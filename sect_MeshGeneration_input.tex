\subsection{Input: Assumptions on course triangulation $\T_0$.}
\label{sec:mesh:input}
The initial mesh $\T_0$ is a \emph{regular triangulation} of $\Omega\subset\mathbb{R}^n$ into closed triangles in the sense that two distinct closed-element domains are either disjoint or their intersection is one common vertex or one common edge. We suppose that each element with domain in $\T_0$ has at least one vertex in the interior of $\Omega$.\\
Given any $T\in\T_0$, one chooses one of its edges $E(T)$ as a \emph{reference edge} such that the following holds. An element $T\in\T_0$ is called isolated if $E(T)$ either belongs to the boundary or equals the side of another element $K\in T_0$ with $E(T)=\partial T\cap\partial K\neq E(K)$. Given a regular triangulation $\T_0$, Algorithm~2.1 of~\cite{CC2004} computes the reference edges ($E(T):T\in\T_0$) such that two distinct isolated triangles do not share an edge. This is important for the $H^1$ stability of the $L^2$ projection in subsection~\ref{sec:Mesh:Properties}.
