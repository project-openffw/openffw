
\subsection{Closure}
In the process of generating adaptive meshed you have to
be careful that the angles of the element are bounded 
due to the maximum angle condition. To guarantee this you
additionally have to refine all reference edges of elements
that have marked edges.
This is done in the function closure.

\medskip
\noindent
\textbf{file:} \path{.\algorithms\misc\closure.m}\\[1.5ex]
\begin{tabular}{@{} l l}
\multicolumn{2}{@{} l}{\textbf{function p = closure(p)}} \\
\textbf{Input:}  & p - ffw\\
\textbf{Output:} & p - ffw\\
\end{tabular}

\medskip
\noindent
We briefly say that although the following code
can have quadratic runtime it has lineare runtime
in average. From convergence theory we know that for
a sequence of triangulations the number of the additionally refined 
reference edges are linear in the number of levels.
Therefor we know that it is in average constant at each
level. Because of that the while loop will be called
in average for a constant number of times. 
\begin{pcode}
I =  refineEdges(ed4e(:,2)) | refineEdges(ed4e(:,3));
while nnz(refineEdges(refEd4e(I))) < nnz(I);
   refineEdges(refEd4e(I)) = true;
   I =  refineEdges(ed4e(:,2)) | refineEdges(ed4e(:,3));
end
\end{pcode}

