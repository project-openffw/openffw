\subsection{Refine}

\textbf{file:} \path{.\algorithms\refine\redGreenBlue.m}\\[1.5ex]
\begin{tabular}{@{} l l}
\multicolumn{2}{@{} l}{\textbf{function p = redGreenBlue(p)}} \\
\textbf{Input:}  & p - ffw\\
\textbf{Output:} & p - ffw\\
\end{tabular}

\medskip
\noindent
At first we create the new c4n. Therefore we take the old c4n 
and add the new coordinates at the end of the list. The new 
coordinates are the midpoints of the marked edges.
%
\begin{pcode}
% Create new node numbers from refineEdges
newNode4ed = zeros(1,nrEdges);
newNode4ed( find(refineEdges) ) = ...
     (nrNodes+1):(nrNodes+nnz(refineEdges));
% Create coordinates of the new nodes
[dontUse,J,S] = find(newNode4ed);
c4n(S,:) = midPoint4ed(J,:);
\end{pcode}
%
In the next step the new n4e is build. At first we calculate the 
number of marked edges for each element. All the Elements that will 
not be refined, e.g. have no marked edge, are fist copied to the new n4e.
In the following new elements will be appended to the list.
%
\begin{pcode}
newNode4e = newNode4ed(ed4e);
unrefinedElems = find( all(newNode4e == 0 ,2) );
refineElems = find( any(newNode4e,2) );
nrMarkedEd4MarkedElems = sum(refineEdges( ed4e(refineElems,:) ),2);
newn4e = n4e(unrefinedElems,:);
\end{pcode}
%
In general there are three different cases to refine an element. 
Elements with one marked edge are refined \emph{green}, elements with 
two marked edges are refined \emph{blue} and elements with tree 
marked edges are refined \emph{red}. To refine all the elements
that are to be refined \emph{green} can be refined simultaneously.
It is important to know that the first edge of an element is 
always the reference edge and that all elements have math. positive 
orientation, i.e., counter clockwise. Because of that fact we know that 
for an element that will be \emph{green} refined the one marked edge is 
the reference edge and therefore the first edge of an element. 
From that fact we see that there is only one way to refine an element \emph{green}. 
Instead we have two different cases with \emph{blue} refinement. 
Therefore we distinguish between \emph{blueleft} and \emph{blueright}
refinement. Again there is only one way to perform \emph{red} refinement
(cf. figure~\ref{f:3}).

\begin{figure}[p]
\begin{center}
\setlength{\unitlength}{4cm}
\begin{picture}(2.5,1.1)

%red
\put(0,0){\line(1,0){1}}
\put(0,0){\line(1,2){0.5}}
\put(1,0){\line(-1,2){0.5}}
\put(0.25,0.5){\line(1,0){0.5}}
\put(0.5,0){\line(-1,2){0.25}}
\put(0.5,0){\line(1,2){0.25}}
\put(0.25,0.5){\circle*{0.05}}
\put(0.5,0){\circle*{0.05}}
\put(0.75,0.5){\circle*{0.05}}
\put(0,0.75){\emph{red}}
\put(-0.04,-0.05){1}
\put(0.48,1.01){3}
\put(1.01,-0.05){2}
\put(0.4,-0.1){$new_1$}
\put(0,0.5){$new_3$}
\put(0.8,0.5){$new_2$}
%\put(0.5,0.6){\vector(0,-1){0.1}}
%\put(0.25,-0.1){\vector(0,1){0.1}}
%\put(0.75,-0.1){\vector(0,1){0.1}}
\put(0.35,0.55){\line(1,0){0.3}}
\put(0.35,0.45){\line(1,0){0.3}}
\put(0.1,0.05){\line(1,0){0.3}}
\put(0.6,0.05){\line(1,0){0.3}}

%green
\put(1.5,0){\line(1,0){1}}
\put(1.5,0){\line(1,2){0.5}}
\put(2.5,0){\line(-1,2){0.5}}
\put(2,0){\line(0,1){1}}
\put(2,0){\circle*{0.05}}
\put(1.5,0.75){\emph{green}}
\put(1.46,-0.05){1}
\put(1.98,1.01){3}
\put(2.51,-0.05){2}
\put(1.9,-0.1){$new_1$}
%\put(2.35,0.55){\vector(-2,-1){0.1}}
%\put(1.66,0.55){\vector(2,-1){0.1}}
\put(1.6,0.1){\line(1,2){0.3}}
\put(2.4,0.1){\line(-1,2){0.3}}
\end{picture}
\vspace{1cm}

\begin{picture}(2.5,1)

%blue left
\put(0,0){\line(1,0){1}}
\put(0,0){\line(1,2){0.5}}
\put(1,0){\line(-1,2){0.5}}
\put(0.5,0){\line(0,1){1}}
\put(0.5,0){\line(-1,2){0.25}}
\put(0.25,0.5){\circle*{0.05}}
\put(0.5,0){\circle*{0.05}}
\put(0,0.75){\emph{blue left}}
\put(-0.04,-0.05){1}
\put(0.48,1.01){3}
\put(1.01,-0.05){2}
\put(0.4,-0.1){$new_1$}
\put(0,0.5){$new_2$}
%\put(0.6,0.5){\vector(-1,0){0.1}}
%\put(0.25,-0.1){\vector(0,1){0.1}}
%\put(0.85,0.55){\vector(-2,-1){0.1}}
\put(0.9,0.1){\line(-1,2){0.3}}
\put(0.1,0.05){\line(1,0){0.3}}
\put(0.45,0.25){\line(0,1){0.5}}

%blue right
\put(1.5,0){\line(1,0){1}}
\put(1.5,0){\line(1,2){0.5}}
\put(2.5,0){\line(-1,2){0.5}}
\put(2,0){\line(0,1){1}}
\put(2,0){\line(1,2){0.25}}
\put(2.25,0.5){\circle*{0.05}}
\put(2,0){\circle*{0.05}}
\put(1.4,0.75){\emph{blue right}}
\put(1.46,-0.05){1}
\put(1.98,1.01){3}
\put(2.51,-0.05){2}
\put(1.9,-0.1){$new_1$}
\put(2.3,0.5){$new_2$}
%\put(1.9,0.5){\vector(1,0){0.1}}
%\put(2.25,-0.1){\vector(0,1){0.1}}
%\put(1.66,0.55){\vector(2,-1){0.1}}
\put(1.6,0.1){\line(1,2){0.3}}
\put(2.1,0.05){\line(1,0){0.3}}
\put(2.05,0.25){\line(0,1){0.5}}

\end{picture}
\end{center}

\caption{\label{f:3} \emph{Red}, \emph{green} and \emph{blue} refinement.
         The new reference edge is marked with a small edge.}
\end{figure}


\noindent
All elements that are marked in refineElemsBisec5 are refined instead of
\emph{red} with \emph{bisec5} (cf. figure~\ref{f:4}).

\begin{figure}[p]
\setlength{\unitlength}{4cm}
\begin{center}
\begin{picture}(1,1.1)
%bisec5
\put(0,0){\line(1,0){1}}
\put(0,0){\line(1,2){0.5}}
\put(1,0){\line(-1,2){0.5}}
\put(0.25,0.5){\line(1,0){0.5}}
\put(0.5,0){\line(-1,2){0.25}}
\put(0.5,0){\line(1,2){0.25}}
\put(0.5,0){\line(0,1){1}}
\put(0.25,0.5){\circle*{0.05}}
\put(0.5,0){\circle*{0.05}}
\put(0.75,0.5){\circle*{0.05}}
\put(0.5,0.5){\circle*{0.05}}
\put(-0.5,0.75){\emph{bisec5}}
\put(-0.04,-0.05){1}
\put(0.48,1.01){3}
\put(1.01,-0.05){2}
\put(0.4,-0.1){$new_1$}
\put(0,0.5){$new_3$}
\put(0.8,0.5){$new_2$}
\put(0.5,0.55){$new_4$}
%\put(0.25,-0.1){\vector(0,1){0.1}}
%\put(0.75,-0.1){\vector(0,1){0.1}}
%\put(0.73,0.2){\vector(-2,1){0.1}}
%\put(0.27,0.2){\vector(2,1){0.1}}
%\put(0.74,0.8){\vector(-2,-1){0.1}}
%\put(0.26,0.8){\vector(2,-1){0.1}}
\put(0.325,0.55){\line(1,2){0.1}}
\put(0.675,0.55){\line(-1,2){0.1}}
\put(0.1,0.05){\line(1,0){0.3}}
\put(0.6,0.05){\line(1,0){0.3}}
\put(0.45,0.175){\line(-1,2){0.1}}
\put(0.55,0.175){\line(1,2){0.1}}
\end{picture}
\vspace{1ex}
\end{center}
\caption{\label{f:4} \emph{bisec5} refinement.}
\end{figure}

\noindent
For example the matlab code for \emph{green} refinement is
printed below. The new elements are 
$T_1=\conv(2,3,new_1)$ and $T_1=\conv(3,1,new_1)$.
The new reference edge are the edges between the nodes
$2,3$ and $3,1$.
\begin{pcode}
I = find(nrMarkedEd4MarkedElems == 1);
if ~isempty(I)
  gElems = refineElems(I);
  [dontUse,dontUse,newN] = find( newNode4e(gElems,:)' );
  newGreenElems = [n4e(gElems,[2 3]) newN;...
                   n4e(gElems,[3 1]) newN];
  newn4e = [newn4e;newGreenElems];
end
\end{pcode}
%
At the end the lists for the boundary, Db and Nb, are
updated.
\begin{pcode}
Db = updateBoundary(Db,DbEd,newNode4ed);
Nb = updateBoundary(Nb,NbEd,newNode4ed);
...
function newBoundary = updateBoundary(oldB,ed4b,newNode4ed)
if(isempty(oldB))
    newBoundary = [];
else
  unrefinedEd = find(~newNode4ed(ed4b));
  refineEd = find(newNode4ed(ed4b));
  newBoundary = [oldB(unrefinedEd,:);...
    oldB(refineEd,1) newNode4ed(ed4b(refineEd))' ;...
    newNode4ed(ed4b(refineEd))' oldB(refineEd,2)];
end
\end{pcode}
